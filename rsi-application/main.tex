\documentclass[10pt, letterpaper]{article}

% Packages:
\usepackage[
    ignoreheadfoot, % set margins without considering header and footer
    top=2 cm, % seperation between body and page edge from the top
    bottom=2 cm, % seperation between body and page edge from the bottom
    left=2 cm, % seperation between body and page edge from the left
    right=2 cm, % seperation between body and page edge from the right
    footskip=1.0 cm, % seperation between body and footer
    % showframe % for debugging 
]{geometry} % for adjusting page geometry
\usepackage{titlesec} % for customizing section titles
\usepackage{tabularx} % for making tables with fixed width columns
\usepackage{array} % tabularx requires this
\usepackage[dvipsnames]{xcolor} % for coloring text
\definecolor{primaryColor}{RGB}{0, 79, 144} % define primary color
\usepackage{enumitem} % for customizing lists
\usepackage{fontawesome5} % for using icons
\usepackage{amsmath} % for math
\usepackage[
    pdftitle={Keshav Anand's RSI Application},
    pdfauthor={Keshav Anand},
    pdfcreator={LaTeX with RenderCV},
    colorlinks=true,
    urlcolor=primaryColor
]{hyperref} % for links, metadata and bookmarks
\usepackage[pscoord]{eso-pic} % for floating text on the page
\usepackage{calc} % for calculating lengths
\usepackage{bookmark} % for bookmarks
\usepackage{lastpage} % for getting the total number of pages
\usepackage{changepage} % for one column entries (adjustwidth environment)
\usepackage{paracol} % for two and three column entries
\usepackage{ifthen} % for conditional statements
\usepackage{needspace} % for avoiding page brake right after the section title
\usepackage{iftex} % check if engine is pdflatex, xetex or luatex
\usepackage{xstring}


% Ensure that generate pdf is machine readable/ATS parsable:
\ifPDFTeX
    \input{glyphtounicode}
    \pdfgentounicode=1
    % \usepackage[T1]{fontenc} % this breaks sb2nov
    \usepackage[utf8]{inputenc}
    \usepackage{lmodern}
\fi



% Some settings:
\AtBeginEnvironment{adjustwidth}{\partopsep0pt} % remove space before adjustwidth environment
\pagestyle{empty} % no header or footer
\setcounter{secnumdepth}{0} % no section numbering
\setlength{\parindent}{0pt} % no indentation
\setlength{\topskip}{0pt} % no top skip
\setlength{\columnsep}{0cm} % set column seperation
\makeatletter
\let\ps@customFooterStyle\ps@plain % Copy the plain style to customFooterStyle
\patchcmd{\ps@customFooterStyle}{\thepage}{
    \color{gray}\textit{\small Keshav Anand - Page \thepage{} of \pageref*{LastPage}}
}{}{} % replace number by desired string
\makeatother
\pagestyle{customFooterStyle}

\titleformat{\section}{\needspace{4\baselineskip}\bfseries\large}{}{0pt}{}[\vspace{1pt}\titlerule]

\titlespacing{\section}{
    % left space:
    -1pt
}{
    % top space:
    0.3 cm
}{
    % bottom space:
    0.2 cm
} % section title spacing

\renewcommand\labelitemi{$\circ$} % custom bullet points
\newenvironment{highlights}{
    \begin{itemize}[
        topsep=0.10 cm,
        parsep=0.10 cm,
        partopsep=0pt,
        itemsep=0pt,
        leftmargin=0.4 cm + 10pt
    ]
}{
    \end{itemize}
} % new environment for highlights

\newenvironment{highlightsforbulletentries}{
    \begin{itemize}[
        topsep=0.10 cm,
        parsep=0.10 cm,
        partopsep=0pt,
        itemsep=0pt,
        leftmargin=10pt
    ]
}{
    \end{itemize}
} % new environment for highlights for bullet entries


\newenvironment{onecolentry}{
    \begin{adjustwidth}{
        0.2 cm + 0.00001 cm
    }{
        0.2 cm + 0.00001 cm
    }
}{
    \end{adjustwidth}
} % new environment for one column entries

\newenvironment{twocolentry}[2][]{
    \onecolentry
    \def\secondColumn{#2}
    \setcolumnwidth{\fill, 4.5 cm}
    \begin{paracol}{2}
}{
    \switchcolumn \raggedleft \secondColumn
    \end{paracol}
    \endonecolentry
} % new environment for two column entries

\newenvironment{header}{
    \setlength{\topsep}{0pt}\par\kern\topsep\centering\linespread{1.5}
}{
    \par\kern\topsep
} % new environment for the header

\newcommand{\placelastupdatedtext}{% \placetextbox{<horizontal pos>}{<vertical pos>}{<stuff>}
  \AddToShipoutPictureFG*{% Add <stuff> to current page foreground
    \put(
        \LenToUnit{\paperwidth-2 cm-0.2 cm+0.05cm},
        \LenToUnit{\paperheight-1.0 cm}
    ){\vtop{{\null}\makebox[0pt][c]{
        \small\color{gray}\textit{Last updated in December 2025}\hspace{\widthof{Last updated in December 2025}}
    }}}%
  }%
}%

% save the original href command in a new command:
\let\hrefWithoutArrow\href

% new command for external links:
\renewcommand{\href}[2]{\hrefWithoutArrow{#1}{\ifthenelse{\equal{#2}{}}{ }{#2 }\raisebox{.15ex}{\footnotesize \faExternalLink*}}}


\begin{document}
    \newcommand{\AND}{\unskip
        \cleaders\copy\ANDbox\hskip\wd\ANDbox
        \ignorespaces
    }
    \newsavebox\ANDbox
    \sbox\ANDbox{}

    \placelastupdatedtext
    \begin{header}
        \textbf{\fontsize{24 pt}{24 pt}\selectfont Keshav Anand — RSI Application}

    %     \vspace{0.3 cm}

    %     \normalsize
    %     \mbox{{\color{black}\footnotesize\faMapMarker*}\hspace*{0.13cm}Your Location}%
    %     \kern 0.25 cm%
    %     \AND%
    %     \kern 0.25 cm%
    %     \mbox{\hrefWithoutArrow{mailto:youremail@yourdomain.com}{\color{black}{\footnotesize\faEnvelope[regular]}\hspace*{0.13cm}youremail@yourdomain.com}}%
    %     \kern 0.25 cm%
    %     \AND%
    %     \kern 0.25 cm%
    %     \mbox{\hrefWithoutArrow{tel:+90-541-999-99-99}{\color{black}{\footnotesize\faPhone*}\hspace*{0.13cm}0541 999 99 99}}%
    %     \kern 0.25 cm%
    %     \AND%
    %     \kern 0.25 cm%
    %     \mbox{\hrefWithoutArrow{https://yourwebsite.com/}{\color{black}{\footnotesize\faLink}\hspace*{0.13cm}yourwebsite.com}}%
    %     \kern 0.25 cm%
    %     \AND%
    %     \kern 0.25 cm%
    %     \mbox{\hrefWithoutArrow{https://linkedin.com/in/yourusername}{\color{black}{\footnotesize\faLinkedinIn}\hspace*{0.13cm}yourusername}}%
    %     \kern 0.25 cm%
    %     \AND%
    %     \kern 0.25 cm%
    %     \mbox{\hrefWithoutArrow{https://github.com/yourusername}{\color{black}{\footnotesize\faGithub}\hspace*{0.13cm}yourusername}}%

    \end{header}

    \vspace{0.2 cm}


    \section{1. Why did you choose these research fields?}

        \vspace{0.2 cm}


        
        \begin{onecolentry}
            \textbf{Prompt: }Articulate why the research fields chosen on the previous page are intriguing and exciting to you. For each sub-field, state what you perceive as the one or two most interesting questions or problems in this area.  Explain why these sorts of questions interest you. Your responses are shared with mentors. Please respond with clarity and specificity, including what specific prior research/coursework/etc experiences have prepared you to ``hit the ground running'' in these fields at RSI.

        \end{onecolentry}

        \vspace{0.2 cm}

        \begin{onecolentry}
            \begin{highlights}
                \item Field 1: Computer Science — Machine Learning for Signal Processing
                \item Field 2: Robotics/Mechatronics — Autonomous Motion Planning
            \end{highlights}
        \end{onecolentry}

        \vspace{0.2 cm}


        \begin{onecolentry}
            \textbf{Limit: 5000 Characters}

        \end{onecolentry}


        \vspace{0.2 cm}

         \newcommand{\mytext}{Firstly, I find Computer Science deeply intriguing as it is the perfect crossover between my two passions of
            mathematics and problem-solving. The ability to optimize a computer chip to solve a pratical problem has 
            always fascinated me, and the rising popularity of Machine Learning and Artificial Intelligence further
            piqued my interest during the COVID-19 pandemic. My research project from 2024 - 2025, GaitGuardian, started my journey
            with signal processing, as I had worked on a project to predict Freezing of Gait (FoG) episodes in Parkinson's
            Disease patients using a belt-mounted IMU sensor and machine learning algorithms. In addition to building
            an end-to-end hardware embedding with a custom PCB, I also developed a strong understanding of signal processing
            techniques. My novel pipeline involved using fourier transforms, z-score normalization, and wavelet denoising
            to filter out noise from the raw IMU data. Unlike existing approachs that used time-domain features, I fed the
            cleaned time-series data into a 1D CNN, acting as an automatic feature extractor (with no flattening). 
            This was passed into a hybrid biLISTM with temporal and spatial attention mechanisms,
            allowing for segmented windows to be read both forwards and backwards, and a final dense layer would
            output the boolean state of whether a FoG episode was occuring in real time. The learning I gained from this research led 
            me to pursue other related signal processing tasks to boost the final product for Parkinson's patients. 
            Researching other Parkinsonian symptoms led me to explore tremor detection (uncontroleld shaking of the hands), 
            and I implemented a real-time tremor detection model that involved a bandpass butterworth filter to isolate tremor frequencies
            between 4-6 Hz, followed by an FFT to extract frequency-domain features. These features were then fed into a lightweight
            1D CNN, resulting in state-of-the-art 99\% accuracy while limiting false positives. I also looked into signal processing 
            within my FTC robotics team, realizing that IMU data could be used to improve odometry and localization. I implemented a 
            custom Kalman filter to fuse IMU data with wheel encoder reading, significantly reducing drift during autonomous navigation 
            and reducing error buildup over time. }



        \begin{onecolentry}


        \StrLen{\mytext}[\n]
        \mytext{ }(\n)

        \end{onecolentry}






    
  

 
    

\end{document}
