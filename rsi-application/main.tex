\documentclass[10pt, letterpaper]{article}

% Packages:
\usepackage[
    ignoreheadfoot, % set margins without considering header and footer
    top=2 cm, % seperation between body and page edge from the top
    bottom=2 cm, % seperation between body and page edge from the bottom
    left=2 cm, % seperation between body and page edge from the left
    right=2 cm, % seperation between body and page edge from the right
    footskip=1.0 cm, % seperation between body and footer
    % showframe % for debugging 
]{geometry} % for adjusting page geometry
\usepackage{titlesec} % for customizing section titles
\usepackage{tabularx} % for making tables with fixed width columns
\usepackage{array} % tabularx requires this
\usepackage[dvipsnames]{xcolor} % for coloring text
\definecolor{primaryColor}{RGB}{0, 79, 144} % define primary color
\usepackage{enumitem} % for customizing lists
\usepackage{fontawesome5} % for using icons
\usepackage{amsmath} % for math
\usepackage[
    pdftitle={Keshav Anand's RSI Application},
    pdfauthor={Keshav Anand},
    pdfcreator={LaTeX with RenderCV},
    colorlinks=true,
    urlcolor=primaryColor
]{hyperref} % for links, metadata and bookmarks
\usepackage[pscoord]{eso-pic} % for floating text on the page
\usepackage{calc} % for calculating lengths
\usepackage{bookmark} % for bookmarks
\usepackage{lastpage} % for getting the total number of pages
\usepackage{changepage} % for one column entries (adjustwidth environment)
\usepackage{paracol} % for two and three column entries
\usepackage{ifthen} % for conditional statements
\usepackage{needspace} % for avoiding page brake right after the section title
\usepackage{iftex} % check if engine is pdflatex, xetex or luatex
\usepackage{xstring}


% Ensure that generate pdf is machine readable/ATS parsable:
\ifPDFTeX
    \input{glyphtounicode}
    \pdfgentounicode=1
    % \usepackage[T1]{fontenc} % this breaks sb2nov
    \usepackage[utf8]{inputenc}
    \usepackage{lmodern}
\fi



% Some settings:
\AtBeginEnvironment{adjustwidth}{\partopsep0pt} % remove space before adjustwidth environment
\pagestyle{empty} % no header or footer
\setcounter{secnumdepth}{0} % no section numbering
\setlength{\parindent}{0pt} % no indentation
\setlength{\topskip}{0pt} % no top skip
\setlength{\columnsep}{0cm} % set column seperation
\makeatletter
\let\ps@customFooterStyle\ps@plain % Copy the plain style to customFooterStyle
\patchcmd{\ps@customFooterStyle}{\thepage}{
    \color{gray}\textit{\small Keshav Anand - Page \thepage}
}{}{} % replace number by desired string
\makeatother
\pagestyle{customFooterStyle}

\titleformat{\section}{\needspace{4\baselineskip}\bfseries\large}{}{0pt}{}[\vspace{1pt}\titlerule]

\titlespacing{\section}{
    % left space:
    -1pt
}{
    % top space:
    0.3 cm
}{
    % bottom space:
    0.2 cm
} % section title spacing

\renewcommand\labelitemi{$\circ$} % custom bullet points
\newenvironment{highlights}{
    \begin{itemize}[
        topsep=0.10 cm,
        parsep=0.10 cm,
        partopsep=0pt,
        itemsep=0pt,
        leftmargin=0.4 cm + 10pt
    ]
}{
    \end{itemize}
} % new environment for highlights

\newenvironment{highlightsforbulletentries}{
    \begin{itemize}[
        topsep=0.10 cm,
        parsep=0.10 cm,
        partopsep=0pt,
        itemsep=0pt,
        leftmargin=10pt
    ]
}{
    \end{itemize}
} % new environment for highlights for bullet entries


\newenvironment{onecolentry}{
    \begin{adjustwidth}{
        0.2 cm + 0.00001 cm
    }{
        0.2 cm + 0.00001 cm
    }
}{
    \end{adjustwidth}
} % new environment for one column entries

\newenvironment{twocolentry}[2][]{
    \onecolentry
    \def\secondColumn{#2}
    \setcolumnwidth{\fill, 4.5 cm}
    \begin{paracol}{2}
}{
    \switchcolumn \raggedleft \secondColumn
    \end{paracol}
    \endonecolentry
} % new environment for two column entries

\newenvironment{header}{
    \setlength{\topsep}{0pt}\par\kern\topsep\centering\linespread{1.5}
}{
    \par\kern\topsep
} % new environment for the header

\newcommand{\placelastupdatedtext}{% \placetextbox{<horizontal pos>}{<vertical pos>}{<stuff>}
  \AddToShipoutPictureFG*{% Add <stuff> to current page foreground
    \put(
        \LenToUnit{\paperwidth-2 cm-0.2 cm+0.05cm},
        \LenToUnit{\paperheight-1.0 cm}
    ){\vtop{{\null}\makebox[0pt][c]{
        \small\color{gray}\textit{Last updated in December 2025}\hspace{\widthof{Last updated in December 2025}}
    }}}%
  }%
}%

% save the original href command in a new command:
\let\hrefWithoutArrow\href

% new command for external links:
\renewcommand{\href}[2]{\hrefWithoutArrow{#1}{\ifthenelse{\equal{#2}{}}{ }{#2 }\raisebox{.15ex}{\footnotesize \faExternalLink*}}}


\begin{document}
    \newcommand{\AND}{\unskip
        \cleaders\copy\ANDbox\hskip\wd\ANDbox
        \ignorespaces
    }
    \newsavebox\ANDbox
    \sbox\ANDbox{}

    \placelastupdatedtext
    \begin{header}
        \textbf{\fontsize{24 pt}{24 pt}\selectfont Keshav Anand — RSI Application}

    %     \vspace{0.3 cm}

    %     \normalsize
    %     \mbox{{\color{black}\footnotesize\faMapMarker*}\hspace*{0.13cm}Your Location}%
    %     \kern 0.25 cm%
    %     \AND%
    %     \kern 0.25 cm%
    %     \mbox{\hrefWithoutArrow{mailto:youremail@yourdomain.com}{\color{black}{\footnotesize\faEnvelope[regular]}\hspace*{0.13cm}youremail@yourdomain.com}}%
    %     \kern 0.25 cm%
    %     \AND%
    %     \kern 0.25 cm%
    %     \mbox{\hrefWithoutArrow{tel:+90-541-999-99-99}{\color{black}{\footnotesize\faPhone*}\hspace*{0.13cm}0541 999 99 99}}%
    %     \kern 0.25 cm%
    %     \AND%
    %     \kern 0.25 cm%
    %     \mbox{\hrefWithoutArrow{https://yourwebsite.com/}{\color{black}{\footnotesize\faLink}\hspace*{0.13cm}yourwebsite.com}}%
    %     \kern 0.25 cm%
    %     \AND%
    %     \kern 0.25 cm%
    %     \mbox{\hrefWithoutArrow{https://linkedin.com/in/yourusername}{\color{black}{\footnotesize\faLinkedinIn}\hspace*{0.13cm}yourusername}}%
    %     \kern 0.25 cm%
    %     \AND%
    %     \kern 0.25 cm%
    %     \mbox{\hrefWithoutArrow{https://github.com/yourusername}{\color{black}{\footnotesize\faGithub}\hspace*{0.13cm}yourusername}}%

    \end{header}

    \vspace{0.2 cm}


    \section{1. Why did you choose these research fields?}

        \vspace{0.2 cm}


        
        \begin{onecolentry}
            \textbf{Prompt: }Articulate why the research fields chosen on the previous page are intriguing and exciting to you. For each sub-field, state what you perceive as the one or two most interesting questions or problems in this area.  Explain why these sorts of questions interest you. Your responses are shared with mentors. Please respond with clarity and specificity, including what specific prior research/coursework/etc experiences have prepared you to ``hit the ground running'' in these fields at RSI.

        \end{onecolentry}

        \vspace{0.2 cm}

        \begin{onecolentry}
            \begin{highlights}
                \item Field 1: Computer Science — Machine Learning for Signal Processing
                \item Field 2: Robotics/Mechatronics — Autonomous Motion Planning
            \end{highlights}
        \end{onecolentry}

        \vspace{0.2 cm}


        \begin{onecolentry}
            \textbf{Limit: 5000 Characters}

        \end{onecolentry}


        \vspace{0.2 cm}

         It was during a COVID-19 YouTube binge where I was first introduced to the art of modern 
            computer. After stumbling into a rabbit hole of videos explaining the inner-workings of a machine,
             I immediately fell in love with Computer Science.
            Over the years, I have also developed a specialized interest in signal processing and its applications.
            My curiousity in this topic stems from my first Math Club meeting in 9th grade, where a senior officer
            had explained the fascinating science of how radio signals are transmitted and received using Fourier transforms.
            These concepts overlapped with my learning of fascinating Calculus concepts, and I was amazed at how signals can 
            be analyzed and processed. Today, the major question that excites me within the field of Signal Processing is how to effectively
            adapt digital signal processing and machine learning for real-time resource-constrained embedded systems. 
            As a hardcore robotics enthusiast, I have always been not just interested in the theoretical software,
            but also the practical hardware embedding of these algorithms. This really sparked my interest in the field 
            of signal processing for embedded systems, prompting me to start a 2 year research project that would become the 
            focus of my life. 
            My ISEF-winning research project from 2024 - 2025, GaitGuardian, was my first major experience with signal processing, 
            as I had worked on a project to predict Freezing of Gait (FoG) episodes in Parkinson's
            Disease patients using a belt-mounted IMU sensor and machine learning algorithms. My novel pipeline involved using fourier transforms, z-score normalization, and wavelet denoising
            to filter out noise from the raw IMU data. Unlike existing approachs that used time-domain features, I fed the
            cleaned time-series data into a 1D CNN, acting as an automatic feature extractor (with no flattening). 
            This was passed into a hybrid biLISTM with temporal and spatial attention mechanisms,
            allowing for segmented windows to be read both forwards and backwards, and a final dense layer would
            output the boolean state of whether a FoG episode was occuring in real time. The learning I gained from this research led 
            me to pursue other related signal processing tasks to boost the final product for Parkinson's patients. 
            Researching other Parkinsonian symptoms led me to explore tremor detection (uncontrolled shaking of the hands), 
            and I implemented a real-time tremor detection model that involved a bandpass butterworth filter to isolate tremor frequencies
            between 4-6 Hz, followed by an FFT to extract frequency-domain features. These features were then fed into a lightweight
            1D CNN, resulting in state-of-the-art 99\% accuracy while limiting false positives. I also looked into signal processing 
            within my FTC robotics team, realizing that IMU data could be used to improve odometry and localization. I implemented a 
            custom Kalman filter to fuse IMU data with wheel encoder reading, significantly reducing drift during autonomous navigation 
            and reducing error buildup over time. 
            My second major interest has been in the field of Robotics, springing from a lucky acceptance into my 
            Middle School's robotics team in 6th grade. Due to COVID-19, out team had to start from scratch, and as a 
            completely inexperienced 7th grader, it took me 7 months to simply learn to spin a motor. The same fascination 
            I had with computers was now being applied to physical hardware, and I have been a loyal participant in 
            the First Tech Challenge (FTC) robotics competition. As the software lead of my globally ranked team,
            Technical Turbulence FTC, I have learned a lot about the algorithms that empower robots during the 
            30-second autonomous period of the competition. Today, I am intrigued by two major research questions within 
            the field of autonomous motion planning. First, I wonder how multiple autonomous agents can effectively
            coordinate in real-time to achieve a common goal while avoiding collisions. This question fascinates me
            because it combines elements of path planning, communication protocols, and decision-making under uncertainty.
            Secondly, I am fascinated by the question of whether autonomous robots and vehicles can learn optimal paths from 
            experience rather than relying on pre-programmed maps. This idea of reinforcement learning for motion planning 
            excites me because it provides a pathway for devices to improve performance over time in dynamic environments.
            My experience with robotics has provided me with a strong foundation to tackle these questions, as I have 
            designed and implemented a custom pathing algorithm for my FTC robot. The motion profiling algorithm I developed 
            uses cubic and quintic splines to generate smooth trajectories between points, using inverse kinematics and a PID 
            controller to accurate follow the path. By prioritizing endpoint accuracy over time and path accuracy, our robot's 
            pathing is extermely precise, resulting in a top-30 autonomous ranking globally. Outside FTC, I worked on a passion 
            project to allow for pathing of two vaccuum robots in a shared environment. Using A* for initial pathfinding and
            a custom potential fields algorithm for real-time obstacle avoidance, I made a software system that allowed 
            for efficient cleaning of a dynamic space. Together, my experience in robotics software, signal processing, and 
            machine learning have prrepared me to hit the ground running at RSI, and I am excited to further explore these research questions
            with the help of expert mentors and resources.






    \section{2. What are your long term goals?}



        \begin{onecolentry}
            \textbf{Limit: 5000 Characters}

        \end{onecolentry}


        \vspace{0.3 cm}

         Every word amazed me. ChatGPT did not feel like yet another website so much 
         as a magical portal to infinite knowledge. I still remember the thrill when my computer science teacher 
         unveiled the mystical tool that could ace our rock-paper-scissors coding exam in under a minute. Little did I
         realize that in less than three years, AI would touch everything I do —
         from school projects to software debugging, even revising this very essay. But it was in the Summer of 2023 that 
         my best friend asked me a life-changing question: ``How does it work?'' I was completely clueless, and I told him 
         that a stenographer was typing at the other end. As months passed and my AI usage exponentially increased, 
         this question haunted me. I was determined not to be a mere user, but to understand and develop the magic 
         behind the curtains. 
         \\\\
         That Summer was when I embarked on my first long-term goal: to fully understand the 
         inner workings of the devices and programs I use. My journey in achieving this started when I joined my 
         robotics team, Technical Turbulence. As the sole software member, I learned everything from scratch. 
         I started by dissecting each wire, realizing that understanding hardware was crucial to mastering software. 
        From PWM control to I2C communication, I slowly worked up the layers of abstraction, eventually
        coming to the software that I was responsible to write. Today, I continue my pursuit of deep technical knowledge 
        through my independent research projects. In my ninth grade, I decided to grasp the fascinating concept of a 
        thermoelectric generator in my research. The following year, I completely shifted focus, diving into 
        high-level signal processing
        and machine learning concepts. 
        Through these experiences, I have learned that true mastery comes from understanding the layers beneath the surface,
        which is why I am committed to this lifelong goal of deep technical knowledge. As I continue to uncover more 
        information, I continue pursuing this goal through passion projects. My latest endeavor of hosting a full 
        server on an old chromebook is pushing me to learn Linux system administration, networking, and cybersecurity. The final 
        result is the same as a Replit fork, but the knowledge I gain from understanding the server-side is invaluable. It is with this goal 
        in mind that I approach RSI, eager to learn from experts and deepen my understanding of computer science.
        \\\\
        My Parkinson's research was also started for this very purpose: to learn an understand the complex fields of
        signal processing and machine learning with a senior friend of mine. However, it quickly evolved into a mission 
        to use my limited knowledge to make a real-world impact. During our presentation at the Dallas Science and Engineering 
        Fair, a judge approached us after our presentation and shared that his father suffered from Parkinson's disease. I didn't 
        think much of it at the time, but when we emailed the Dallas Area Parkinson's Society (DAPS) to share our findings,
        the overwhelming response from patients and caregivers made me realize the true potential of our work. Two months later,
        a close family friend was diagnosed with early-onset Parkinson's, and I realized the hope that research brings to people. 
        Therein lies my second long-term goal: use my knowledge to improve the human condition through impactful research. 
        \\\\
        Although my skillset is still limited, my goal is to continue to push the boundaries of technological applications to 
        eventually benefit humanity. Not only am I interested in healthcare applications like GaitGuardian, but I am also fascinated 
        by the potential of robotics and software to improve everyday life. For example, I was particularly piqued by the 
        recent advancements in autonomous pathing of multiple agents, as I see huge potential for applications in warehouse automation, 
        construction, and even self-driving cars. Be it through hardware, software, or a combination of both, I am committed to
        expand my knowledge and use it to make a positive impact on the world. RSI represents a crucial step in this journey, allowing me 
        to work on relevant research that pushes the boundaries of technology. As a devout Hindu, the Vibuthi (cow ash) 
        I apply to my forehead reminds me that all humans will eventually unite with the Earth,
         and I want to be remembered for leaving a positive mark on humanity before I do. 
        \\\\
        My final long-term goal is to encourage and inspire the next generation of students to pursue STEM. As someone who 
        has experienced the transformative power of a good teacher firsthand, I am passionate about helping 
        others discover the joy of learning. After realizing that a teacher can change my perception on a subject itself, 
        I committed myself to tutoring and helping others. Whether it is through robotics outreach or through my school's 
        ACE tutoring club, I have always sought to share my knowledge and enthusiasm for STEM with others. I hope RSI presents me 
        with an opportunity to further this goal by providing me world-class mentorship and resources, which I can then 
        use to benefit others in the same way RSI will benefit me.
        




    
    \section{3. What activities and/or hobbies demonstrate your leadership, creativity and uniqueness?}
        

        \begin{onecolentry}
            \textbf{Limit: 5000 Characters}

        \end{onecolentry}


        \vspace{0.3 cm}


            Music was my first language. Even before I could speak, music became the place where I learned to create.
            Throughout elementary school, I learned Piano and Classical Indian Carnatic vocals, but even after I picked 
            up flute for my middle school band, but I never felt I was truly expressing myself. 
            It was finally in 8th grade, when I decided to drop my perpetual 
            practicing and pursue a form of music that reflected me: Indian film music covers. Unlike Western music, which 
            centers on albums, popular music in the Indian subcontinent is woven into cinema, with most movies 
            featuring five to six full-length songs. I wanted to not just replicate these songs, but to enhance them,
            leaving my own fingerprint on familiar melodies.\\\\
            Here, my creativity finally blossomed in the form of tasteful covers that completely
            reimagined familiar melodies. After learning about the world of Digital Audio Workstations (DAWs), I realized that 
            most musical instruments can be reproduced with a digital keyboard language known as MIDI. This completely changed 
            my life — propelling me into a universe that lies at the intersection of my two greatest passions: music and 
            technology. I spent countless hours tinkering with plugins and virtual soundtracks to digitally capture 
            the subtle beauty of each sound. I even bought a \$50 Black Friday bass guitar and began 
            merging analog and digital sounds. My music started to feel like me. \\\\
            Joining my first band, High Octavez, was transformative. 
            As a group of hobbyist musicians dedicated to faithfully recreating iconic film music, 
            High Octavez taught me what it mean to collaborate a a high level.
            Together, we recreated iconic film music with precision, and all proceeds went to charity. 
            As the keyboard player, I didn’t just perform; I sculpted the sounds, blending technical skill with 
            creative vision. Performing in two concerts, which collectively raised over \$300,000 for charity,
            showed me how creativity can make tangible impact. Not only was I expressing myself, I was 
            helping my community. Now, as I prepare for my third concert, I am continuing to fuse music and 
            technology. Using my server-side knowledge to develop a music streaming service, I am able to
            express my creativity in a positive manner by combining my talents and passions.
            \\\\
            As much as I had collaborated in High Octavez, I would collaborate more in my studies. From 
            the moment I entered high school, my friends and I formed a joint study circle where 
            we would discuss upcoming assignments and tests. Before we learned the biological concept of 
            mutualism, we were practicing it through 10 PM Discord calls. Midway through ninth grade, I added 
            another friend to our group chat...then two. Within a matter of 30 minutes, we had invited over 
            30 people to join our newly made study server for the Plano East Class of 2027. Before long, 
            rules had to be set, and as the server owner, I had made it clear that no swearing would be 
            allowed. For obvious reasons, this did not sit well with my peers, and moderation quickly became 
            a nightmare. For rogue non-Plano East students invading the server to a perpetual spew of racial 
            slurs, we were quickly overwhelmed. Every loophole was penetrated; every vulnerability was exploited.
            At that exact moment, I began using my coding skills to our advantage, developing Discord bots to 
            manage moderation. I began outsourcing work to other moderators and study guides, leading 
            a group of nearly 200 members to create a haven for studious students
             Before long, all members required school ID-based verification, and I had
            integrated AI into moderation to prevent penetration. At the end of my 10th grade, nearly 60%
            of my class had joined the server. By hosting study sessions and posting valuable studying 
            resources, my server had gained the trust of not just students but teachers alike. In fact, 
            our human geography teacher hosted a last-minute AP exam study session through the server.



            %Add LASER

    \section{4. Describe your participation in extracurricular or community outreach activities? }


        \begin{onecolentry}
            \textbf{Limit: 5000 Characters}

        \end{onecolentry}


        \vspace{0.3 cm}

        % NHS, Cricket Club, LASER?

        My heart was pounding relentlessly: every second felt like an eternity. It was my first robotics 
        competition, and my code would determine our team's advancement. As the delay continues, my mind 
        races with all the possible errors. Finally, the announcer begins his countdown, and all I can think of 
        is the robot crashing into the wall, ending our competition. The buzzer sounds, I press play — nothing. 
        No movement, the robot simply stays in place. As I watched the program fail over the next 
        thirty seconds, all I felt was pure embarrassment. Losing in the final round, I promised myself that I would 
        my code would never perform so poorly again. \\\\
        And so it began, my alcoholic addition towards the First Tech Challenge (FTC). Entering high school, 
        I switched from my middle school team to a small community team consisting of the same crew from 
        Schimelpfenig Robotics. At this point, I was an experienced programmer in FTC, but my actual skillset 
        matched that of a grape. I knew no programming languages, I couldn't understand high-level concepts, and 
        I was yet to learn even 5\% of the mathematics needed for autonomous navigation. I was thrown into the 
        deep end of the pool, and our team had to stay afloat. \\\\
        I learned. I learned the basics of the Java 
        programming language, and I had joined the school's CS Club to better my skills. Before I knew it, 
        I was able to make our robot fully move and score points using only Java. As my team's sole programmer, 
        I slowly learned how to use sensors, motors, and servos. Progress was finally made. Our team had mustered 
        enough capability to become a state finalist, and we were finally excelling. Entering my sophomore year, 
        I became a relatively decent programmer, and our team had expanded to include other programmers. That Summer, 
        I took on a challenge to code an autonomous pathing system, and after 4 months of constant iteration, 
        my code could finally move a robot from point to point accurately. Leading our software unit, I had 
        expanded our autonomous and driver operated capabilities to a world-class level, ranking in the top 30 globally 
        in autonomous rank. Today, as I continue to spend countless hours on robotics, I am using my knowledge to 
        expand our team's capabilities. Having learned multivariable calculus and qualifying for the AIME, my math 
        skills were finally being applied to my passion of robotics. Hopefully, our team can break our final barrier,
        and win the world championship this year.
        \\\\
        Everything in my life had perfectly modeled the stereotypical nerd: poor posture, obsession with CS, and 
        having more screen time on my calculator. Three months ago, however, I finally broke this mold 
        by founding my school's cricket club. After a wrist fracture had brought my competitive career to a complete 
        standstill, I was determined to I was determined to get back into the sport I loved somehow.
        When I realized that a school with nearly 600 Desi students had no cricket team, I started 
        the Plano East Cricket Club with a few friends. Unlike the competitive teams I had played in, 
        this club would play with a softer taped-tennis ball, reducing the need for equipment (and costs). 
        At first, the club started as a way to rally the cricket-players within the school, but a lack of attendance 
        opened the club towards complete beginners. It proved a daunting task to not only play, but to also teach 
        the sport as a semi-coach, and I was learning new coaching methods to keep practices engaging. As we play our 
        games, we have always been severe underdogs (and lose terribly), but our team's growth from complete scratch
        gives me a sense of closure in continuing the sport I suddenly left.   \\\\
        In addition to helping my school community learn cricket, I have also sought to help the larger community
        through my school's National Honor Society (NHS). In a competitive application process, I was fortunate enough to be
        selected as the NHS technology officer for the largest chapter in the world. In this role, I have sought to 
        modernize our chapter's operations through technology. By developing and maintaining a custom hours logging 
        portal, I have streamlined the process of tracking service hours for over 1300 members. By combining my 
        coding skills with my passion for service, I have been able to make a tangible impact on my community. 
        As we continue to develop QR-code based check-ins and AI-powered event recommendations, I am excited to further
         enhance our NHS operations through technology.



         

    
  

 
    

\end{document}
