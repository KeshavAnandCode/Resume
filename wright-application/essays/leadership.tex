\begin{enumerate}
	\item I am the founder and president of the Cricket Club at my current high school. I identified a gap in our school's athletic program and built the Cricket Club into a sustainable organization with 25+ active members. As founder, I recruited members through Instagram campaigns and word-of-mouth outreach. I organized and ran biweekly practice sessions, acting as both player and coach to guide new members through the sport's steep learning curve. Each practice required coordinating schedules across 25+ students, securing field space, managing equipment, and ensuring everyone stayed engaged. As a first-year club, I worked to bridge the skill gap between experienced cricketers and complete beginners, ensuring everyone was challenged while still having fun. I also organized matches with other local high school teams, coordinating with their club leaders, arranging transportation, and managing all match-day logistics.

	\item I serve as Co-Captain and Software Lead for my school's First Tech Challenge (FTC) Robotics team. As the sole programmer on our initially rookie team, I developed the entire codebase for our competition robot from scratch. When funding became a critical obstacle, I secured a \$750 sponsorship from Texas Instruments through targeted outreach. As our team expanded, I recruited new members, strategically selecting students based on our programming and engineering needs. Once we grew to five programmers, I restructured our development workflow by creating a GitHub organization with multiple repositories for collaborative experimentation and version control. I delegated tasks and managed project deadlines to keep our software development on schedule. For competitions, I led documentation of our software iterations and technical innovations, which contributed to winning the Innovate and Control awards.

	\item I serve as Vice President (and former underclassmen president) of my school's Science Fair Club. As an officer, I lead biweekly lunch meetings for 90+ active members, collaborating with fellow officers to design engaging activities and delivering 35-40 minute presentations on critical aspects of science fair projects. I have developed comprehensive instructional materials covering project brainstorming, methodology, data analysis, and trifold design to guide students through the competition process. Beyond meetings, I provide one-on-one mentorship via email and in-person consultations, helping students navigate project decisions based on my own competition experience. Drawing from my initial intimidation with the science fair process, I specifically design presentations to lower the barrier to entry for new students and underclassmen, making a traditionally daunting competition more accessible.

	\item From halfway through my 9th grade to now, I have founded, owned, and maintained a Discord study server for my school's class of 2027. As owner, I manage a team of eight elected moderators and facilitate democratic decision-making to ensure all members' voices are heard. Because disagreements can escalate quickly online, I regularly serve as a neutral mediator to de-escalate and resolve conflicts between members. I have dedicated over 300 hours to managing this server, which has supported 100+ active members and, at its peak, engaged over 50\% of our class. I have navigated challenges ranging from minor disputes to major conflicts, including attempts to remove me as owner, requiring diplomacy and conflict resolution under pressure. Today, the server is a stable and thriving community offering curated study resources, regular group study calls, and consistent engagement through academic and social channels.

	\item I serve as the Chemistry Officer (and former Math Officer) for my school's tutoring club, ACE. My primary role is to supervise student volunteers who create review materials for Chemistry, delegating units and topics to contributors and reviewing all posted resources to ensure accuracy and appropriateness. I also organize and monitor live tutoring sessions for AP Chemistry, both in-person and via Zoom, serving 20-30 students per session. My most demanding responsibility is stepping in to complete any unfinished materials or host sessions when other volunteers are unavailable, requiring me to maintain flexible availability to ensure a seamless experience for students seeking academic support.

	\item I serve as the Technology Officer for my school's National Honor Society chapter. In addition to my technical duties, I moderate service events for 200+ student volunteers, managing both participant behavior and the execution of service activities to ensure productive community impact. I also represent our chapter at leadership conferences such as LEAD, where I collaborate with officers from other NHS chapters to develop and share innovative leadership strategies.

	\item In my freshman year, I served as an officer for my school's Math Club. As an officer, I presented approximately one-quarter of our weekly sessions to 15+ students, designing and delivering engaging lessons on mathematical applications and concepts such as mathematical card tricks and calculator programming. I was also responsible for maintaining detailed attendance records for each meeting to track participation requirements for the Mu Alpha Theta honor society.
\end{enumerate}