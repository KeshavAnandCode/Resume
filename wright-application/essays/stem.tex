% FTC, TEG, GaitGuardian, AIME + Contributing Solutions to AOPS, ACE, Full Server Setup, Hacking?, Science Bowl?, Debate Congress APp, MIDI technology??, 

In the last three years, STEM activities have been the focus of my high school experience, both through and outside of school.
I have had a strong passion towards STEM activities since middle school, and I believe that these experiences have prepared me
to fully utilize the research platforms and opportunities provided by AFRL.

\begin{enumerate}
	\item From 9th to 11th grade, I have been an active member of my community First Tech Challenge (FTC) Robotics team, serving as
	      Co-Captain and Software Lead. I started in my freshman year from square one, knowing no programming languages or
	      concepts. Slowly, I picked up Java through the FTC SDK, and I eventually started writing functional code for our robot. As the
	      season had progressed, I had become a decent programmer, and our team was able to win as a State Finalist alliance team and
	      Area Innovate Award Winner for a unique fully-automated hang mechanism. In the following two years, I ramped up my
	      time commitment and progressed my skills. By applying the Calculus knowledge I was learning in the classroom, I was
	      able to implement a fully custom autonomous pathing system for our robot, using trigonometry, inverse kinematics, and
	      PID control to achieve precise movement. In addition, I tinkered with computer vision, developing a custom TensorFlow model
	      for failsafe object detection (with 100\% match success). Our team went on to champion our league, become an Area Finalist
	      captain, and place in the global top-30 for autonomous performance.
	\item In my 9th grade, my independent research project for Science Fair qualified as a finalist project for the
	      International Science and Engineering Fair (ISEF). My project started from a very simple idea of utilizing available thermal energy
	      in a cooking pot to also stir the contents (to save energy for cooking). From here, I learned about the world of
	      thermoelectric generators (TEGs) and the Seebeck effect. Applying the CAD and simulation concepts I learned from my
	      Engineering class in school, I designed an aluminum enclosure that used a TEG, a heat sink, and a motor to convert heat
	      energy into mechanical stirring. I also learned about electrical circuits as I used a voltage and current sensor to effectively 
          graph my TEG's performance relative to the temperature differential. My final prototype, although simple, was able to win 
          1st in Engineering Technology at my regional science fair, where I also won a special awards through the US Metric Association and 
          the US Air Force Certificate of Recognition. At ISEF, I was able to present to domain experts and professors from 
          around the world, gaining valuable insights and feedback to later improve my prototype and project.
          \item 


\end{enumerate}