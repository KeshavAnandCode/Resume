% FTC, TEG, GaitGuardian, AIME + Contributing Solutions to AOPS, ACE, Full Server Setup, Hacking?, Science Bowl?, Debate Congress APp, MIDI technology??, 

In the last three years, STEM activities have been the focus of my high school experience, both through and outside of school.
I have had a strong passion towards STEM activities since middle school, and I believe that these experiences have prepared me
to fully utilize the research platforms and opportunities provided by AFRL.

\begin{enumerate}
	\item From 9th to 11th grade, I served as Co-Captain and Software Lead for my community's First Tech Challenge (FTC) Robotics team. Starting freshman year with no programming experience, I taught myself Java through the FTC SDK and progressed to developing fully functional robot code that contributed to our team reaching the State Finals and winning the Area Innovate Award for a fully automated hang mechanism. Over the following two years, I significantly deepened my technical contributions. By applying calculus concepts from the classroom, I engineered a custom autonomous pathing system using trigonometry, inverse kinematics, and PID control for precise robot movement. I also developed a custom TensorFlow object detection model, achieving a 100\% match success rate across all competitions. These innovations helped our team win our league championship, captain an Area Finalist alliance, and rank in the global top 30 for autonomous performance.

	\item In 9th grade, my independent Science Fair research project qualified as a finalist at the International Science and Engineering Fair (ISEF). The project originated from a simple idea: using waste thermal energy from a cooking pot to power a mechanical stirring system. Through this process, I learned about thermoelectric generators (TEGs) and the Seebeck effect. Applying CAD and simulation concepts from my engineering coursework, I designed an aluminum enclosure integrating a TEG, heat sink, and motor to convert heat energy into mechanical motion. I also developed an electrical measurement system using voltage and current sensors to analyze performance relative to temperature differentials. My prototype won 1st place in Engineering Technology at my regional science fair, along with special awards from the U.S. Metric Association and the U.S. Air Force Certificate of Recognition. At ISEF, I presented my work to professors and domain experts from around the world and received valuable feedback to improve my design.

	\item In 10th grade, my Science Fair research project earned 3rd place in Robotics and Intelligent Machines at the International Science and Engineering Fair (ISEF). Motivated by relatives affected by Parkinson’s Disease, I explored machine learning approaches to assist with gait and tremor analysis. Using publicly available datasets, I designed a novel signal processing algorithm capable of accurately classifying gait patterns with low inference latency. After training and testing multiple models, I expanded the project by creating a physical prototype. With guidance from my engineering teacher, I designed a custom PCB featuring an ESP32 microcontroller and an IMU sensor capable of performing real-time inference. Through independent study of academic literature and online tutorials, I taught myself Python and C++. Initially a multiple regional winner, the project won 1st place in Systems Software, the TI Best Computing Project Award, and 2nd Grand Prize, qualifying me for ISEF for a second year. At ISEF, it earned 3rd place in Robotics and Intelligent Machines, along with a \$1{,}200 award.

	\item This year, I maintained a full-time home server by repurposing a decade-old laptop running Ubuntu Linux. Through hands-on experimentation, I self-learned bash scripting, networking, Linux administration, and service management. I deployed a Matrix messaging server for student communication, a Git server for code hosting, and an SSH service with a custom interactive shell displaying my portfolio. Despite more than one million automated attacks within two months, my hardened security measures prevented intrusion attempts, providing practical experience in networking, cybersecurity, and infrastructure management.

	\item In 9th grade, I qualified for the American Invitational Mathematics Examination (AIME) through the AMC 10 competition. Through self-study using online resources, I developed strong problem-solving skills focused on logical reasoning, speed, and optimization. I participated in and placed at multiple math competitions, including Purple Comet and Math League. I also contribute solutions on the Art of Problem Solving forum using \LaTeX{} to clearly communicate my reasoning. Additionally, I create Geometry and Calculus problems for use in my school’s ACE tutoring club.

	\item During my 10th grade, I developed a real-time audio visualization program for live musical performances. I built a Python pipeline that converts MIDI input to audio using a piano VST, then processes the signal through the Librosa library to extract and visualize frequency content via Fast Fourier Transforms. By leveraging MIDI’s low-latency input, the system achieves near-instant visual response, enabling live performers to generate professional concert visuals without expensive hardware or post-production software.

	\item During my 10th-grade year, I developed a web-based utility application for debate tournaments using React and TypeScript to address operational inefficiencies. I designed precise timing software with graphical notifications and audio alerts, as well as an automated presiding officer system using the WebSocket API to synchronize moderators and competitors in real time. The application has been adopted and positively reviewed by national finalists and members of my school’s debate team.

	\item Over the past three years, I completed four online courses covering Java, Bash, and Python. Because I did not receive formal instruction in programming languages, these courses were instrumental in enabling me to independently complete and extend my technical projects.

	\item From 9th to 11th grade, I consistently practiced competitive programming by solving LeetCode and USACO problems in Java. Through this problem-oriented approach, I strengthened my understanding of arrays, recursion, data structures, and time complexity optimization.

	\item In the fall of my 11th grade, I completed ethical hacking challenges on Hack The Box (HTB). Using a Kali Linux virtual machine, I gained root access on vulnerable machines, learning core cybersecurity concepts including enumeration, establishing footholds, and privilege escalation.
\end{enumerate}