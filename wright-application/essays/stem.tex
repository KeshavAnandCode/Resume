% FTC, TEG, GaitGuardian, AIME + Contributing Solutions to AOPS, ACE, Full Server Setup, Hacking?, Science Bowl?, Debate Congress APp, MIDI technology??, 

In the last three years, STEM activities have been the focus of my high school experience, both through and outside of school.
I have had a strong passion towards STEM activities since middle school, and I believe that these experiences have prepared me
to fully utilize the research platforms and opportunities provided by AFRL.

\begin{enumerate}
	\item From 9th to 11th grade, I have been an active member of my community First Tech Challenge (FTC) Robotics team, serving as
	      Co-Captain and Software Lead. I started in my freshman year from square one, knowing no programming languages or
	      concepts. Slowly, I picked up Java through the FTC SDK, and I eventually started writing functional code for our robot. As the
	      season had progressed, I had become a decent programmer, and our team was able to win as a State Finalist alliance team and
	      Area Innovate Award Winner for a unique fully-automated hang mechanism. In the following two years, I ramped up my
	      time commitment and progressed my skills. By applying the Calculus knowledge I was learning in the classroom, I was
	      able to implement a fully custom autonomous pathing system for our robot, using trigonometry, inverse kinematics, and
	      PID control to achieve precise movement. In addition, I tinkered with computer vision, developing a custom TensorFlow model
	      for failsafe object detection (with 100\% match success). Our team went on to champion our league, become an Area Finalist
	      captain, and place in the global top-30 for autonomous performance.
	\item In my 9th grade, my independent research project for Science Fair qualified as a finalist project for the prestigious
	      International Science and Engineering Fair (ISEF). My project started from a very simple idea of utilizing available thermal energy
	      in a cooking pot to also stir the contents (to save energy for cooking). From here, I learned about the world of
	      thermoelectric generators (TEGs) and the Seebeck effect. Applying the CAD and simulation concepts I learned from my
	      Engineering class in school, I designed an aluminum enclosure that used a TEG, a heat sink, and a motor to convert heat
	      energy into mechanical stirring. I also learned about electrical circuits as I used a voltage and current sensor to effectively
	      graph my TEG's performance relative to the temperature differential. My final prototype, although simple, was able to win
	      1st in Engineering Technology at my regional science fair, where I also won a special awards through the US Metric Association and
	      the US Air Force Certificate of Recognition. At ISEF, I was able to present to domain experts and professors from
	      around the world, gaining valuable insights and feedback to later improve my prototype and project.
	\item In my 10th grade, my research project for Science Fair won 3rd in Robotics and Intelligent Machines at the
	      highly-competitive International Science and Engineering Fair (ISEF).
	      After seeing many relatives of mine struggle with Parkinson's Disease, I wanted to help create a solution that could help them.
	      Through research, I encountered an interesting method of using machine learning to aid in both walking and tremors. Using
	      online published datasets, I applied a novel online signal processing approach that ensured real-time classification of
	      gait patterns. After training multiple models and testing them through cross-validation, I ended up with a fully functional model
	      with high accuracy and low inference times. However, I wanted to take this a step further and create a physical prototype. Using
	      my engineering teacher's help, I designed a custom PCB with an ESP32 microcontroller and an IMU sensor, which I programmed
	      to collect real time data to feed to my model. By using online resources and reading published literature methods, I was able to
	      self-teach myself the needed Python and C++ to fully implement my working solution. My novel approach was recognized at the regional level,
	      where I won 1st in Systems Software, TI Best Computing Project, and 2nd Grand Prize (qualifying me to ISEF). At ISEF, domain experts were
	      equally impressed with my work, and I was awarded 3rd place in Robotics and Intelligent Machines, along with a \$1200 award.
	\item This year, I have started a passion project in maintaining a full-time server. After repurposing a decade-old budget laptop,
	      I installed an Ubuntu Linux OS onto my machine and setup a home server. From here, I slowly the basics of bash scripting,
	      networking, Linux, and service management through a hands-on approach. I set up multiple services, including a
	      Matrix client used by my friends for communicating while in school. I also set up a Git server to host my code,
	      which taught me a lot about permissions and how servers actually interact with clients. Most recently, I set up an
	      SSH service which allows users to view a shell interactive I made to display my portfolio. While this extremely risky
	      ,with over 1 million attacks in two months, I have learned a lot about security measured, and none of the attacks have
	      been successful due to my proactive measures. While my server is mostly used for tinkering, I have learned a lot about
	      networking and computing fundamentals through this project.
	\item In my 9th Grade, I started taking on competition math to challenge myself, qualifying for the AIME (American Invitational
	      Mathematics Examination) through AMC 10. After self-studying through online resources and textbooks, I learned the art of
	      solving math problems in a intuitive, creative, and timely manner. Not only did I learn advanced math concepts, but I also
	      learned the critical logical thinking and problem-solving skills needed to tackle complex problems. I have also
	      competed and placed in local math competitions, including Purple Comet and Math League. As I encounter these problems
	      through the online community Art of Problem Solving (AoPS), I have also started contributing solutions to problems.
	      I learned how to clearly and concisely explain my thought process and solutions in LaTeX format,
	      helping other students understand my unique approach to these complex problems. Within my high school, I also
	      contribute challenging problems through our tutoring club, ACE, to help students in Geometry and Calculus.
	      %//TODO: SHORTEN BOTH ABOVE??
	\item % MIDI AND AUDIO
	\item % Debate congrews app
	\item % Self LEarnign courses and LEet code, etc.??
\end{enumerate}