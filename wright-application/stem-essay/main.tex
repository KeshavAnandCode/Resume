\documentclass[12pt, letterpaper]{article}

% Packages:
\usepackage[
    top=0.5in,
    bottom=0.75in,
    left=0.75in,
    right=0.75in,
    headheight=0pt,
    headsep=0pt,
    footskip=0.4in
]{geometry}
\usepackage{times} % Times New Roman font
\usepackage[utf8]{inputenc}
\usepackage[T1]{fontenc}
\usepackage{setspace} % for single spacing
\usepackage{titlesec} % for customizing section titles
\usepackage[dvipsnames]{xcolor}
\definecolor{primaryColor}{RGB}{0, 79, 144}
\usepackage[
    pdftitle={Wright Scholar Essay},
    pdfauthor={Keshav Anand},
    pdfcreator={LaTeX},
    colorlinks=false,
    hidelinks
]{hyperref}
\usepackage{iftex}
\usepackage{microtype} % Better text rendering

% Ensure PDF is machine readable:
\ifPDFTeX
    \input{glyphtounicode}
    \pdfgentounicode=1
\fi

% Settings:
\pagestyle{empty} % no header or footer
\setlength{\parindent}{0.5in} % standard paragraph indentation
\setlength{\parskip}{6pt} % no space between paragraphs
\singlespacing % single spacing
\frenchspacing % Better spacing after periods

% Custom title format for essay topic
\titleformat{\section}
    {\normalfont\fontsize{12}{14.4}\selectfont\bfseries}
    {}
    {0pt}
    {}
\titlespacing{\section}{0pt}{0pt}{6pt}

\begin{document}

\begin{center}
	\textbf{\large Wright Scholar Essay (Topics 1 and 3)}
\end{center}



\vspace{12pt}

294 Squiggly red underlines. Nearly every line of my code had errors.
Null pointers, incompatible types, undefined variables, Gradle sync errors---I
had encountered them all. It was February 2024, my freshman year, and we had
ten minutes to take the field for our First Tech Challenge (FTC) League Finals.
My heart pounded as keys clattered beneath my flying fingers. My code was broken,
and for the finals, it had to work. The merciless clock ticked away, and with
seconds to go, I finally compiled the code. There was no time to test, hardly
any to breathe. We took the field, and my finger hovered over the play button.
Time paused. The buzzer sounded, and I pressed play. Success. In two minutes
and thirty seconds, we won.

Seven months earlier, I didn't know what a variable was. I was fully into music,
and programming wasn't even on my radar. When my friend started a robotics team,
I joined on a whim. My journey began with a Google search. Progress was
painstakingly slow; it took me two full months to make a motor turn. But
gradually, I became hooked. Like a sponge, I absorbed everything: tutorials,
documentation, and even Stack Overflow threads. Eventually, I taught myself
enough Java to become a functional FTC programmer.

As the season progressed, we became a competitive team, and my knowledge was
expanding in parallel. On that competition day, something just clicked. The
joy I experienced wasn't just from our robot picking up and scoring pixels,
but from seeing my code produce tangible results. In that moment, I'd found
my calling. I was no longer just a high school student; I was a STEM student,
and I was ready to see where my code could take me.

But that readiness was tested in September 2024. Somewhat naively, I committed
to building a machine learning model to predict gait patterns in Parkinson's
Disease for my sophomore-year Science Fair project. The problem? I had no clue
how. So I dove in: Python syntax, NumPy arrays, signal filtering, feature
extraction, and model architectures. I had entered unfamiliar territory, and
each concept brought new confusion. After two months of relentless reading,
coding, and debugging, I managed to transform raw sensor data into a working
classification model. Somewhere between the first error message and the final
96\% accuracy, I had begun to absorb a new discipline.

I could have stopped there, but I realized that a working model on my laptop
wasn't going to help any Parkinson's patients, and I needed to embed my model
into a complete hardware device. This task was beyond daunting, as I had to
venture into the foreign territory of hardware and electrical engineering.
With my engineering teacher guiding me, I eventually learned the basics. After
countless 2 AM KiCAD tutorial sessions, I finally had a working design for a
custom printed circuit board (PCB) housing a sensor and microcontroller. Two
weeks later, my PCB arrived, and after soldering all my components, it didn't
work. My heart sank. In desperation, I resoldered each joint carefully and
tried again. Somehow, it worked. After writing some C++ software for the
device, I had something that actually worked. The project eventually made it
to the International Science and Engineering Fair (ISEF), placing 3rd in
Robotics and Intelligent Machines. What struck me most wasn't the placement,
but the fact that six months earlier, I wouldn't have understood any of it.

Throughout high school, I've taught myself disciplines, from Java programming
to machine learning to circuit design. The Wright Scholar program offers an
opportunity to apply my knowledge to critical research. I'm drawn to AFRL's
Sensors Directorate, where I hope to deepen my understanding of signal
processing while contributing to sensor exploitation technologies. I'm equally
fascinated by the Human Performance Wing's work with multimodal sensing to
monitor and enhance human performance. What excites me most isn't just the
cutting-edge technology, but the chance to work alongside domain experts who
can accelerate my growth as an engineer and developer. Whether working with
sensor fusion or biomedical sensing, as a sponge eager to learn, AFRL is
exactly where I need to be.



\end{document}