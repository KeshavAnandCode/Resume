\documentclass[12pt, letterpaper]{article}

% Packages:
\usepackage[
    top=0.5in,
    bottom=0.75in,
    left=0.75in,
    right=0.75in,
    headheight=0pt,
    headsep=0pt,
    footskip=0.4in
]{geometry}
\usepackage{times} % Times New Roman font
\usepackage[utf8]{inputenc}
\usepackage[T1]{fontenc}
\usepackage{setspace} % for single spacing
\usepackage{titlesec} % for customizing section titles
\usepackage[dvipsnames]{xcolor}
\definecolor{primaryColor}{RGB}{0, 79, 144}
\usepackage[
    pdftitle={Wright Scholar Essay},
    pdfauthor={Keshav Anand},
    pdfcreator={LaTeX},
    colorlinks=false,
    hidelinks
]{hyperref}
\usepackage{iftex}
\usepackage{microtype} % Better text rendering

% Ensure PDF is machine readable:
\ifPDFTeX
    \input{glyphtounicode}
    \pdfgentounicode=1
\fi

% Settings:
\pagestyle{empty} % no header or footer
\setlength{\parindent}{0.5in} % standard paragraph indentation
\setlength{\parskip}{0pt} % no space between paragraphs
\singlespacing % single spacing
\frenchspacing % Better spacing after periods

% Custom title format for essay topic
\titleformat{\section}
    {\normalfont\fontsize{12}{14.4}\selectfont\bfseries}
    {}
    {0pt}
    {}
\titlespacing{\section}{0pt}{0pt}{6pt}

\begin{document}


\noindent\textbf{Topic 3:} Research involves trying new things. Please give an example of when you stepped out of your comfort zone to experience something new in regards to science or engineering.

\noindent\textbf{Topic 4:} What do you hope to get out of the Wright Scholar experience (be creative)?

\vspace{12pt}

% Essay content begins here
294 squiggly red underlines. Nearly every line of my code had errors. Null pointers,
incompatible types, undefined variables, Gradle sync errors---I had seen it all, and each error felt like a
dissonant chord demanding resolution. It was February 2024, my freshman year, and we had ten minutes to
take the field for our First Tech Challenge (FTC) final match. My heart raced as I thumped the keyboard in a
frenzy. My code was broken, and for the final match, it had to work. Time raced faster than it ever used to, and
I finally compiled the code.
There was no time to test, hardly any to breathe, and before I knew it, we were on the field with my index
finger hovering over the large play button. Time paused. I heard the buzzer and pressed play. Success.
In two minutes and thirty seconds, we became league champions.

It was almost hard to believe that seven months back, I didn't know what a variable was.
I was fully into music, and programming was not even an afterthought.
It was mere coincidence that my neighbor (and good friend) decided to start a robotics team, and given the minimal investment, I
joined. Like nearly all of my endeavors, my FTC journey began with a Google search. I was learning at a snail's pace, and it
had taken me two months to simply make a motor move. Soon, I was hooked. Like a sponge, I was absorbing everything I had to learn,
and I had eventually taught myself enough Java to become a functional FTC programmer.

As the season progressed, my sponge was
soaked, and we were a top competitive team by February. On competition day, the lightbulb within me finally clicked.
The joy I experienced wasn't just from our robot picking up and scoring pixels, but from the fact that code I had recently learn to write
was resulting in a tangible output that I could witness. It was that moment where I decided to pursue a STEM career. I was
no longer just a high school student, I was a STEM student, and I was ready to help change the world.

But that readiness was tested in September 2024. In a spur of ambitious insanity,
I had committed to building a machine learning model to predict gait patterns in Parkinson's Disease for my sophomore-year
Science Fair project. The problem: I had no clue how to. And so I learned. Python syntax, NumPy arrays, signal filtering,
feature extraction, model architectures. I had entered a brand new domain, and each concept seemed to confuse me in a different way.
After two months of painfully laborious learning, coding, and debugging, I was finally able to transform raw sensor data into a
functional and accurate classification model. Somewhere between the first error message and the final 96\% accuracy, I
had managed to absorb a new discipline by pushing myself into unfamiliar waters.



\end{document}